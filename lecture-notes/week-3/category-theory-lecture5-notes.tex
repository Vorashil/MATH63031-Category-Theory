\documentclass{article}
\usepackage[utf8]{inputenc}

\usepackage{tikz}
\usepackage{tikz-cd}
\usepackage{graphicx} % Required for inserting images
\usepackage[a4paper, margin=1in]{geometry}
\usepackage{amsfonts}
\usepackage{amsmath} % Set custom margins
\usepackage{parskip}
\usepackage{amssymb} % disable indentation
\usepackage{amsthm}
\usepackage{annotate-equations}


\title{Category Theory - Lecture 5 (Notes)}
\author{Vorashil Farzaliyev}
\date{September 2024}

% Define a new theorem style that starts content on a new line


\newtheorem{remark}{Remark}[section]
\newtheorem{lemma}{Lemma}[section]
\newtheorem{proposition}{Proposition}[section]
\newtheorem{definition}{Definition}[section]
\newtheorem{observation}{Observation}[section]
\newtheorem{example}{Example}[section]
\newtheorem{exercise}{Exercise}[section]
\newtheorem{theorem}{Theorem}[section]
\newtheorem{note}{Note}[section]

\renewcommand{\shorto}{\to}
% new command for making \mapsto 0.5cm

\renewcommand{\to}{\xrightarrow{\hspace{0.5cm}}}  % Adjust the length as needed
\renewcommand{\mapsto}{\mathrel{\kernel\longmapsto\kernel}}  % Adjust spacing with \mkern if needed

\begin{document}
    \maketitle

    \begin{exercise}
        Let $\mathcal{C}$ be a category.

        Let \(f, g: A \to B\) and let \(u: A' \to A\) be an isomorphism.
        Then,
        \[
            f = g \Leftrightarrow f \circ u = g \circ u
        \]
    \end{exercise}

    \begin{proof}
        \((\Rightarrow)\): It is clear.


        \((\Leftarrow)\): If \(u: A' \to A\) is an isomorphism, then there exists \(u^{-1}: A \to A'\) such that

        \[
            \begin{align}
                fu = gu &\Rightarrow fuu^{-1} = guu^{-1} \\
                &\Rightarrow f = g \\
            \end{align}
        \]
    \end{proof}


    \section{Natural Transformations}

    \begin{definition}
        Let \(F, G: \mathbf{C} \to \mathbf{D}\) be functors.

        A \textbf{natural transformation} \(\phi: F \Rightarrow G\) is a family of maps:

        \[
            \{\phi_A: FA \to GA \ | \ A \in Ob(\mathbf{C})\}
        \]

        such that for all \(f: A \to B\) in \(\mathbf{C}\), we have

        \[
            \begin{array}{c c} % Create a 2x2 array
                \mathbf{C} & \mathbf{D} \\
                \begin{tikzcd}[column sep=4em, row sep=4em]
                    A \arrow[d, mapsto, "f"] \\
                    B
                \end{tikzcd}
                &
                \begin{tikzcd}[column sep=7em, row sep=4em]
                    FA \arrow[r, "\phi_A"] \arrow[d, "f"]
                    & GA \arrow[d, "Gf"] \\
                    FB\arrow[r, "\phi_B"]
                    & GB
                \end{tikzcd}
            \end{array}
        \]
    \end{definition}


    \vspace{0.2in}

    \newpage

    \begin{example}
        Fix \(n \in \mathbb{N}\)

        \begin{itemize}
            \item \(\mathbf{C} = \underline{CRing} = \) category of commutative rings
            \item \(\mathbf{D} = \) category of monoids
            \item \(F: \underline{CRing} \to \underline{Mon}\) is the functor
            \[
                F = M_n: \underline{CRing} \to \underline{Mon}
            \]
            Here \(M_n(R)\) is \(n \times n\) matrix with entries in \(R\). So \(\underline{Mon}\) is
            the category of monoids with \(n \times n\) matrices as objects.
            \[
                \begin{tikzcd}[column sep=7em, row sep=4em]
                    R \arrow[r, mapsto,  "F"] \arrow[d, "f"]
                    & M_n(R) \arrow[d, "M_n(f)"] \\
                    S \arrow[r, "F"]
                    & M_n(S)
                \end{tikzcd}
            \]
            \item \(G: \underline{CRing} \to \underline{Mon}\) is the functor
            \[
                G = U: \underline{CRing} \to \underline{Mon}
            \]
            Here we have
            \[
                \begin{tikzcd}[column sep=7em, row sep=4em]
                (R,+, *, 0, 1)
                    \arrow[r, mapsto,  "U"] \arrow[d, "f"]
                    & (R, +, 1) \arrow[d, "U(f)"] \\
                    (S,+, *, 0, 1) \arrow[r, "U"]
                    & (S, *, 1)
                \end{tikzcd}
            \]
            So \(U\) simply returns the underlying monoid of a given ring.
        \end{itemize}

        Now we can define a natural transformation

        \[
            det : M_n \Longrightarrow U
        \]

        For \(R = (R, +, *, 0, 1) \in \underline{CRing}\), we need

        \begin{gather*}
            det_R: FR \to GR\\
            det_R: M_n(R) \to U(R)\\
            det_R: M_n(R) \to (R, *, 1)\\
        \end{gather*}

        Here \(det_R\) is a monoid homomorphism

        This is monoid homomorphism because

        \[
            \left\{
            \begin{array}{ll}
                det_R(M * N) = det_R(M) * det_R(N) \text{ (it preserves multiplication)} \\
                \vspace{0.1in}                                                           \\
                det_R(I_n) = 1_R   \text{ (it preserves identity)}                          \\
            \end{array}
            \right.
        \]

        So now need to check that this is a natural transformation. (i.e we need to check its \textbf{naturality}).
        \[
            \begin{array}{c c} % Create a 2x2 array
                \underline{\mathbf{Cring}} & \underline{\mathbf{Mon}} \\
                \begin{tikzcd}[column sep=4em, row sep=4em]
                    R \arrow[d, mapsto, "f"] \\
                    S
                \end{tikzcd}
                &
                \begin{tikzcd}[column sep=7em, row sep=4em]
                    M_n(R) \arrow[r, "det_R"] \arrow[d, "M_n(f)"]
                    & R \arrow[d, "f"] \\
                    M_n(S) \arrow[r, "det_S"]
                    & S
                \end{tikzcd}
            \end{array}
        \]

        \[
            \begin{tikzcd}[column sep=7em, row sep=4em]
                \begin{pmatrix}
                    a & b \\ c & d
                \end{pmatrix}
                \arrow[r, mapsto,  "det_R"] \arrow[d, "M_n(f)"]
                & ad - bc \arrow[d, "f"] \\
                \begin{pmatrix}
                    fa & fb \\ fc & fd
                \end{pmatrix}
                \arrow[r, "U"]
                & f(ad -bc)
            \end{tikzcd}
        \]

    \end{example}

    \subsection{Notation}

    \subsubsection{Natural transformation diagram}

    For \(F, G: \mathbf{C} \to \mathbf{D}\) and \(\phi: F \Rightarrow G\) as natural transformation, we write

    \[
        \begin{tikzcd}[column sep=7em, row sep=4em]
            \mathbf{C} \arrow[r, bend left=50, "F"{name=F, above}]
            \arrow[r, bend right=50, "G"{name=G, below}]
            & \mathbf{D}
            \arrow[Rightarrow, from=F, to=G, "\phi"]
        \end{tikzcd}
    \]

    \subsubsection{Composition diagram}

    Given \(F, G, H: \mathbf{C} \to \mathbf{D}\) and \(\phi, \psi: F \Rightarrow G\) as natural transformations, we write

    \[
        \begin{tikzcd}[column sep=7em, row sep=4em]
            \mathbf{C} \arrow[r, bend left=50, "F"{name=F, above}]
            \arrow[r, "G"{name=G, above}]
            \arrow[r, bend right=50, "H"{name=H, below}]
            & \mathbf{D}
            \arrow[Rightarrow, from=F, to=G, "\phi"]
            \arrow[Rightarrow, from=G, to=H, "\psi"]
        \end{tikzcd}
    \]

    We define

    \[
        \begin{tikzcd}[column sep=7em, row sep=4em]
            \mathbf{C} \arrow[r, bend left=50, "F"{name=F, above}]
            \arrow[r, bend right=50, "H"{name=H, below}]
            & \mathbf{D}
            \arrow[Rightarrow, from=F, to=H, "\psi\phi"]
        \end{tikzcd}
    \]

    Also

    \[
        FA \stackrel{(\psi\phi)_A}{\xrightarrow{\hspace{1cm}}} HA \ =_{def} \ FA \stackrel{\phi_A}{\to} GA \stackrel{\psi_A}{\to} HA
    \]

    So we can define composition

    \[
        \begin{array}{c c} % Create a 2x2 array
            \mathbf{C} & \mathbf{D} \\
            \begin{tikzcd}[column sep=4em, row sep=4em]
                A \arrow[d, mapsto, "f"] \\
                B
            \end{tikzcd}
            &
            \begin{tikzcd}[column sep=7em, row sep=4em]
                FA \arrow[r, "\phi_A"] \arrow[d, "f"]
                & GA \arrow[d, "Gf"] \arrow[r, "h_A"]
                & HA \arrow[d, "Hf"] \\
                FB\arrow[r, "\phi_B"]
                & GB \arrow[r, "h_B"]
                & HB
            \end{tikzcd}
            % Draw external arrows pointing to the rectangles
            \begin{tikzpicture}[overlay]
                % Arrow to the first rectangle (FA, GA, FB, GB)
                \draw[->, thick] (-5, -3) -- (-5, -0.5) node[near start, auto] {Naturality of \(\phi\)};
                % Arrow to the second rectangle (HA, HB)
                \draw[->, thick] (-2, -3) -- (-2, -0.5) node[near start, right] {Naturality of \(\psi\)};
            \end{tikzpicture}
        \end{array}
    \]

    \vspace{0.2in}

    \subsection{Identity natural transformation}


    For \(F: \mathbf{C} \to \mathbf{D}\), we can define

    \[
        \begin{tikzcd}[column sep=7em, row sep=4em]
            \mathbf{C} \arrow[r, bend left=50, "F"{name=F, above}]
            \arrow[r, bend right=50, "F"{name=G, below}]
            & \mathbf{D}
            \arrow[Rightarrow, from=F, to=G, "1_F"]
        \end{tikzcd}
    \]

    by \(FA \stackrel{1_{FA}}{\xrightarrow{\hspace{1cm}}} FA\) for all \(A \in \mathbf{C}\).

    Naturality of this is

    \[
        \begin{array}{c c} % Create a 2x2 array
            \mathbf{C} & \mathbf{D} \\
            \begin{tikzcd}[column sep=4em, row sep=4em]
                A \arrow[d, mapsto, "f"] \\
                B
            \end{tikzcd}
            &
            \begin{tikzcd}[column sep=7em, row sep=4em]
                FA \arrow[r, "1_{FA}"] \arrow[d, "Ff"]
                & FA \arrow[d, "Ff"] \\
                FB\arrow[r, "1_{FA}"]
                & FB
            \end{tikzcd}
        \end{array}
    \]

    \subsection{Functor category}

    \begin{definition} (Functor category)


        For categories \(\mathbf{C}, \mathbf{D}\), where \(\mathbf{C}\) is a small category, we define the \textbf{functor category} \([\mathbf{C}, \mathbf{D}]\)
        (sometimes denoted as \(\mathbf{D}^{\mathbf{C}}\)) as having

        \begin{itemize}
            \item \textbf{objects}: functors \(F: \mathbf{C} \to \mathbf{D}\), \(G: \mathbf{C} \to \mathbf{D}\)
            \item \textbf{maps}: natural transformations \(\phi: F \Rightarrow G\)
        \end{itemize}
    \end{definition}

    \vspace{0.2in}

    \begin{exercise}
        Check the associativity and unit axioms for functor category.
    \end{exercise}

    \newpage

    \subsubsection{Isomorphisms in \(\mathbf{D}^{\mathbf{C}}\) or \([\mathbf{C}, \mathbf{D}]\)}

    Fix \(\mathbf{C}, \mathbf{D}\).

    By definition, an isomorphism in \([\mathbf{C}, \mathbf{D}]\) is \(\phi: F \Rightarrow G\) with an inverse

    \[
        \phi^{-1}: G \Rightarrow F
    \]

    This means that

    \[
        \begin{tikzcd}[column sep=4em, row sep=4em]
            F  \arrow[Rightarrow, r, "\phi"] \arrow[Rightarrow,dr, bend right, "1_F"]
            & G  \arrow[Rightarrow, d, "\phi^{-1}"]\\
            & F
        \end{tikzcd}
        &
        % Second commutative diagram in the second column
        \begin{tikzcd}[column sep=4em, row sep=4em]
            G  \arrow[Rightarrow, r, "\phi^{-1}"] \arrow[Rightarrow, dr, bend right, "1_G"]
            & F  \arrow[Rightarrow, d, "\phi"]\\
            & G
        \end{tikzcd}
    \]

    This holds if and only if, for all \(A \in \mathbf{C}\), we have

    \[
        \begin{tikzcd}[column sep=4em, row sep=4em]
            FA  \arrow[r, "\phi_A"] \arrow[dr, bend right, "1_{FA}"]
            & GA  \arrow[d, "\phi_A^{-1}"]\\
            & FA
        \end{tikzcd}
        &
        % Second commutative diagram in the second column
        \begin{tikzcd}[column sep=4em, row sep=4em]
            GA  \arrow[r, "\phi_A^{-1}"] \arrow[dr, bend right, "1_{GA}"]
            & FA  \arrow[d, "\phi_A"]\\
            & GA
        \end{tikzcd}
    \]

    This implies \(\forall A \in \mathbf{C}\), \(\phi_A: FA \to GA\) is an isomorphism.

    \vspace{0.5in}

    \begin{proposition}
        Let \(\phi: F \Rightarrow G\) be a natural transformation.

        Then \(\phi\) is an isomorphism if and only if for all \(A \in \mathbf{C}\)
        \[
            \phi_A: FA \to GA
        \]
        is an isomorphism.

    \end{proposition}

    \begin{note}
        We need \(\phi\) to be natural in order to piece together all isomorphisms \(FA \to GA\) for all \(A \in \mathbf{C}\)
    \end{note}

\end{document}
